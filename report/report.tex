\documentclass[11pt, a4paper]{article}

% Packages
\usepackage[francais]{babel}
\usepackage[T1]{fontenc}
\usepackage[utf8]{inputenc}

\usepackage[left=2cm, right=2cm, top=2cm, bottom=2cm]{geometry}
\usepackage{fancyhdr}
\usepackage{lastpage}
\usepackage{hyperref}

\usepackage{float}

\usepackage{graphicx}
\graphicspath{{./img/}}
\usepackage{tikz}

% Reset paragraph indentation -------------------------------------------------
\setlength{\parindent}{0cm}

% Allow a paragraph to have a linebreak ---------------------------------------
\newcommand{\paragraphnl}[1]{\paragraph{#1}\mbox{}\\}

% Page header and footer ------------------------------------------------------
\pagestyle{fancy}
\setlength{\headheight}{33pt}
\renewcommand{\headrulewidth}{0.5pt}
\lhead{\includegraphics[height=1cm]{hepia.jpg}}
\chead{Game of Life}
\rhead{Claudio Sousa - David Gonzalez}
\renewcommand{\footrulewidth}{0.5pt}
\lfoot{11 décembre 2016}
\cfoot{}
\rfoot{Page \thepage /\pageref{LastPage}}

% Table of contents depth -----------------------------------------------------
\setcounter{tocdepth}{3}

% Document --------------------------------------------------------------------
\begin{document}

\title
{
    \Huge{Programmation concurrente} \\
    \Huge{Game of Life}
}
\author
{
    \LARGE{Claudio Sousa - David Gonzalez}
}
\date{11 décembre 2016}
\maketitle

\begin{center}
    Image...%\includegraphics[scale=0.5]{image}
\end{center}

\thispagestyle{empty}

\newpage

% -----------------------------------------------------------------------------
\section{Introduction}

Ce TP de deuxième année consiste à implémenter un Game of Live (jeux de la vie).
La particularité de celui-ci est que tous modules doit s'exécuter en parallèle. \\

Les modules concernés sont:

\begin{itemize}
    \item gestion du clavier (un thread);
    \item gestion d'affichage (un thread);
    \item gestion de la grille du Game of Life (un ou plusieurs threads). \\
\end{itemize}

Le traitement de la grille suit des règles selon 2 paramètres:

\begin{itemize}
    \item l'état de la cellule: morte ou vivante;
    \item le nombre de voisins. \\
\end{itemize}

Ainsi, une cellule vivante meure seulement si elle a 0, 1, ou plus de 3 voisins.

Une cellule morte revit seulement si elle a exactement 3 voisins.

\newpage

% -----------------------------------------------------------------------------
\section{Development}
\subsection{Architecture}

\begin{figure}[H]
    \begin{center}
        \includegraphics[scale=0.65]{gol_modules.png}
    \end{center}
    \caption{Architecture du Game of Life}
    \label{Architecture du Game of Life}
\end{figure}

L'architecture du Game of Life est divisée en 5 modules. \\

Le module \textit{main} est le programme principal.
Il a pour rôle de:

\begin{itemize}
    \item récupérer les paramètres de ligne de commande;
    \item initialiser et lancer les threads des différents modules;
    \item stopper et libérer les modules lorsque demandé. \\
\end{itemize}

Le module \textit{time\_wait} permet simplement a un thread
de se synchroniser avec une fréquence de fonctionnement (en herz). \\

Le module \textit{display} contient le thread qui s'occupe de l'affichage.

Il a la responsabilité d'initialiser la librairie SDL à l'intérieur du thread.
Afin de permettre à d'autre module d'utiliser la SDL, une barrière est utilisée et est joint deux fois:

\begin{itemize}
    \item une fois par le thread d'affiche après l'initialisation de la SDL;
    \item une fois par le thread principal après avoir lancé le thread d'affichage.
\end{itemize}

Ceci permet d'empêcher le thread principal de continuer avant que la SDL ne soit initialisée. \\

Le module \textit{keyboard} est également un thread qui a pour seul rôle de bloquer le thread principal
tant que la touche \textit{ESC} n'a pas été pressée.
Pour cela, le thread principal lance le thread du clavier puis attend sa fin en le joignant immédiatement après l'avoir lancé. \\

Le module \textit{gol} contient l'ensemble de l'algorithme qui permet de traiter la grille en parallèle...

\subsection{Algorithmie}

\subsubsection{Répartition du travail entre les threads du GoL}

\subsubsection{Swap}

\newpage

% -----------------------------------------------------------------------------
\subsection{Concurrence}

\begin{figure}[H]
	\begin{center}
		\includegraphics[width=\textwidth]{workers_flow_diagram.png}
	\end{center}
	\caption{Synchronisation du traitement et de l'affichage}
	\label{Synchronisation du traitement et de l'affichage}
\end{figure}

\subsubsection{Synchronisation du traitement et de l'affichage}

Les flèches à gauche du schéma symbolise les threads, \textit{n} threads pour le traitement de la grille (haut) et le thread pour l'affichage de cette grille (en bas). \\

Tous les threads sont synchornisés à deux endroits, symbolisés par deux barres rouges verticales:
\begin{itemize}
	\item \textit{frame\_done} synchronise tous les threads à la fin du traitement de l'état actuel de la grille.
	\item \textit{frame\_ready} s'assure que tous les threads attendent que la grille suivante soit prête à être traitée.
\end{itemize}

\subsubsection{Condition de sortie}

Lorsque la touche \textit{ESC} est pressée, le thread du clavier se termine et libère le thread principal. Celui-ci met la variable \textit{request\_finish} à \textit{true} et attend que tous les threads se termine. \\

Les threads du traitement de la grille vérifient cette variable avant la barrière \textit{frame\_done} et, le cas échéant, mettent une autre variable \textit{finished} à \textit{true}. Après \textit{frame\_done}, chaque thread vérifie l'état de cette variable et se termine s'elle est mise à \textit{true}. \\

L'utilisaton de deux variables, et surtout la separation entre l'écriture et la lecture de \textit{finish}, permet de s'assurer que sa valeur sera la même pour tous les threads entre \textit{frame\_done} et \textit{frame\_ready}. \\

La fin des threads redonne la main au thread principal qui libère les ressources du programme.

\newpage

% -----------------------------------------------------------------------------
\subsection{Méthodologie de travail}
\subsubsection{Répartition du travail}

\newpage

\end{document}
